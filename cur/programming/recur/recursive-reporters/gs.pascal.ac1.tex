\documentclass[12pt,raggedbottom,nosectionbreak]{cme}
\usepackage{pft}
\usepackage[texshop]{edc-art}
\usepackage{enumitem}
\usepackage{graphicx}

 
 
\def\mbb#1{\text{\bf #1}}
\draftnotice{PROMYS Seminar.  March 27, 2009}
\copyrightdate{2009}
\title{Squares, Sums, and Related Curiosities}

\begin{document}

%\setcounter{page}{0}
%\ruledhabits
%\maketitle
%\tableofcontents 
\chapter*{Getting Started}

\sidenote [dismemo] {\pointer  Some people call this the ``distinguished member method.''   Bullwinkle is the distinguished member.  You count all the committees that contain Bullwinkle, then count all the committees that don't,and then~\dots .}
\begin{foryou}
You're in a class with a total of five students: Rocky, Bullwinkle, Peabody, Sherman, and (your name here). You want to choose a committee to get the refreshments for the class party. Remember that ``Rocky and Sherman'' is the same committee as ``Sherman and Rocky,''

\begin{problem} \label{fipro}
\begin{enumerate}
 
  \item  How many different committees of three people include Bullwinkle?

   \item   How many different committees of three people \emph{don't} include Bullwinkle?
  
 \item   How many\plsn[-40pt]{dismemo} different committees of three people are there in a group of 5 students?
     
 
\end{enumerate}
\end{problem}
  \end{foryou}




\sidenote [hiyse] {\pointer One idea:  use the result of problem~\ref{fipro} and the distinguished member method.}
\begin{problem}
 \begin{enumerate}
\item  How many 2-person committees can you make from 5 students?
\item  How many 3-person committees can you make from 6 students?\plsn{hiyse}

\end{enumerate}
\end{problem}

\begin{problem}
\begin{enumerate}
\item How many 1 member committees can you form from 100 students?

\item How many 99 member committees can you form from 100 students?

\item How many 100 member committees can you form from 100 students?


\sidenote [fiaway] {\pointer How can you have a committee with~0 members?  Find a way for that to make sense.}


\item How many 0 member committees can you form from 100 students?\plsn{fiaway}

\end{enumerate}
\end{problem}


\begin{problem}
Describe the ``distinguished member method'' in words, so precisely that others in your class understand it.
\end{problem}


\begin{problem}  Here's a start on a reporter that will return the number of $k$-person committees that can be formed from $n$-students.
\begin{center}

\includegraphics[scale=.6]{binco}


\end{center}
Finish it off.
\end{problem}








\end{document}